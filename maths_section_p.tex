\documentclass[a4paper,10pt]{article}
% Call packages
\usepackage[utf8x]{inputenc}
\usepackage{float}
\usepackage{placeins}
\usepackage{natbib}
\usepackage{cite}
\usepackage{amstext}
\usepackage{amssymb}
\usepackage{amsmath}
\usepackage{epsfig}
\usepackage{graphics}
\usepackage{graphicx}

% Definitions
\DeclareMathOperator\erf {erf}
\newcommand{\heavi}{\Theta}

% Title
\title{Development of a Radiating Model for the Influence on Tropospheric Response of the Characteristics of Heat Forcing }
\author{O.J. Halliday, S. D. Griffiths, D. J. Parker}

\begin{document}
\maketitle
\begin{abstract}
Maths section after SDG edits
\end{abstract}

\section{Mathematical Model}
\label{sec_Model}
%
From linear governing equations we derive and solve an equation for field variable $w$, 
using a modal expansion, for a sensible heat forcing with defined spatial and temporal variation.
We then obtain the corresponding potential temperature response, $b$. 
Our solution radiates energy at the tropopause and assumes a compressible atmosphere.

Consider small disturbances about a state of rest in an compressible fluid in two dimensions, $(x, z)$,
within a hydrostatic Boussinesq-like approximation:
%
\begin{eqnarray}
\label{equ_Boussinesq}
\frac{\partial u}{\partial t}  = - \frac{1}{\rho_0} \frac{\partial p}{\partial x}, \quad
\frac{g \rho}{\rho_0}  =  - \frac{1}{\rho_0}  \frac{\partial p}{\partial z}, \quad
\frac{\partial \rho}{\partial t} +  w \frac{d \rho_0}{dz}  =  -Q, \quad
\frac{\partial u}{\partial x} + \frac{\partial w}{\partial z}  =  0, 
\end{eqnarray}
%
where $(u, w)$ is the perturbation wind vector, $\rho$ the density, $p$ the pressure, $Q$ the thermal forcing and 
the basic state of density is:
%
\begin{equation}
\rho_0 (z) = \rho_s \exp \left( - \frac{z}{D} \right).
\end{equation}
%
$w$ is subject to the boundary conditions:
%
\begin{eqnarray}
\label{bcs}
w(z = 0, H) = 0. 
\end{eqnarray}

Note that the the thermodynamic equation from equations \ref{equ_Boussinesq} may also be written in terms of 
a buoyancy perturbation as follows (PB):
%
\begin{eqnarray}
\label{equ_b}
\frac{\partial b }{\partial t} + N^2 = S, \quad S = - \frac{g }{ \rho_0 } Q,
\end{eqnarray}
%
where $b = \frac{g \theta'}{\theta_0}$ is a buoyancy perturbation, $\theta'$ a potential temperature perturbation, 
$\theta_0$ a reference potential temperature and $N^2 =-\frac{g}{\rho_0}\frac{d \bar{\rho_0}}{dz}$ is the buyancy frequency 
(see below).

Eliminating variables from equations \ref{equ_Boussinesq} the relationship between  $w$ and $S$ may be obatined:
%
\begin{equation}
\label{equ_vertical_structure}
\frac{1}{\rho_0} \frac{\partial}{\partial z} \left( \rho_0(z)  \frac{\partial}{\partial z} \frac{\partial^2 w}{\partial t^2} \right)  +N^2 \frac{\partial^2 w}{\partial x^2} = \frac{\partial^2 S}{\partial x^2}.
\end{equation}
%
To solve equation \ref{equ_vertical_structure} for a given heating function $S(x,z,t)$ let us first address the vertical variation of the of the $w$ response. 
We shall use a modal expansion of fields $w$ and $S$ between rigid lower and upper boundaries at
 $z=0,H$, with the troposphere (which later will be taken to coincide with the top of heating) corresponding to the restricted range of altitudes:
%
\begin{equation}
0 \leq z \leq H_t, \quad H_t \ll H.
\end{equation}
% 
Similar to equation \ref{equ_vertical_structure} is the following eigenvlaue equation for the free modes:
%
\begin{equation}
\label{equ_free_modes}
\frac{d}{dz} \left( \rho_0(z) \frac{d \phi_n}{dz} \right) + \frac{N^2 \rho_0(x)}{c_n^2} \phi_n(z) = 0,
\end{equation}
%
with associated boudary conditions:
%
\begin{equation}
\phi_n(0) = \phi_n(H) = 0.
\end{equation}
%
With the above boundary conditions we find solutions and corresponding wavespeeds as follows:
%
\begin{equation}
\phi_n(z) = A \exp \left( \frac{z}{H_t} \right) \sin \left( \frac{n \pi}{H} z\right), \quad c_n = \sqrt{  \frac{4 H^2 H_t^2 N^2 }{H^2 + 4 n^2 \pi^2 H_t^2} }.
\end{equation}
%
These solutions, when used with an approriate metric, have the orthomormaility property:
%
\begin{equation}
\int_0^H\rho_0(z)\phi_n(z) \phi_m(z) dz = \delta_{nm}
\end{equation}
%
provided the normalization constant $A$ is chosen as:
%
\begin{equation}
A = \sqrt{ \frac{2}{ \rho_s H} }.
\end{equation}
%
We shall shortly project the vertical structure of the $w$ response onto the discrete $\phi_n$ above using
expansions for $S$ and $w$:
%
\begin{equation}
\label{equ_modal}
w = \sum_i w_i(x,t) \phi_n(z), \quad S = \sum_i s_i(x,t) \phi_n(z). \quad
\end{equation}
%
To avoid differentiating these series, equation \ref{equ_vertical_structure} must first be transformed. 
Multiply by $\phi_m(z)$ and integrate on $z$ in the range $0\leq z\leq H$ to obtain:
%
\begin{equation}
- \int_0^H  \rho_0 \frac{\partial \phi_m}{\partial z}  \frac{\partial}{\partial z} \frac{\partial^2 w}{\partial t^2}  dz  + N^2  \int_0^H \phi_m \rho_0 \frac{\partial^2 w}{\partial x^2} dz  = \int_0^H  \phi_m \rho_0 \frac{\partial^2 S}{\partial x^2} dz.
\end{equation}
%
Here, integration by parts and the assumed boundary conditions on the $\phi_n$ have been on the first term on the left hand side. A second application of parts with
$u= \rho_0 \frac{\partial \phi_m}{\partial z} $ and $\frac{dv}{dz} = \frac{\partial}{\partial z} \frac{\partial^2 w}{\partial t^2} $ and the use of equation \ref{equ_free_modes} 
now yields and equation into which the modal expansions may be substituded:
%
\begin{equation}
- \frac{N_2}{c_n^2} \int_0^H  \rho_0 \phi_m \frac{\partial^2 w}{\partial t^2}  dz  + N^2  \int_0^H \phi_m \rho_0 \frac{\partial^2 w}{\partial x^2} dz  = \int_0^H  \phi_m \rho_0 \frac{\partial^2 S}{\partial x^2} dz.
\end{equation}
%
Substituting the expansions in equation \ref{equ_modal} and using the orhonormaility property of the free modes, we easily obtain an equation for the $w_m$ and $S_m$:
%
\begin{equation}
-  \frac{\partial^2 }{\partial t^2} w_m(x,t)  + c_n^2   \frac{\partial^2 }{\partial x^2} w_m(x,t)   =  \frac{c_n^2 }{N^2} \frac{\partial^2 }{\partial x^2} s_m(x,t).
\end{equation}
%

To proceed, we specify a pulsed or transient thermal forcing of finite duration, $T$, which is of separable form, 
defined over the whole range of vertical co-ordinate:
%
\begin{equation}
\label{equ_heating}
S(x,z,t) = Q F(x) \left( \heavi(t) - \heavi(t-T) \right) G(z).
\end{equation}
%
Here $\heavi(t)$ is the Heavyside function and $Q$ a calibration constant to be used later. We will shortly define function $G(z)$
using it to restrict the vertical range of heating to the troposphere.
%
%
%
\subsection{$w$-Response}
%
We derive the vertical velocity of the response induced by the form of thermal forcing specified in equation \ref{equ_heating}
On making a Fourier transform on horizontal variable $x$, followed by a Laplace transform 
on variable $t$ to \ref{equ_vertical_structure} and using standard properties of the Fourier and Laplace transforms 
 (\citep{arfken2013mathematical}) we obtain:
%
\begin{eqnarray}
\tilde{w}_m(k,p) & = & \frac{Q c_n^2 k^2 \tilde{F}(k) }{ N^2 p \left( p + i c_m \right) \left( p-i c_m \right)  } \\ \nonumber
                        & + &  \frac{Q c_m^2 k^2 \tilde{F}(k)  e^{-pT} }{ N^2 p \left( p + i c_m \right) \left( p-i c_m \right)  }. \\ \nonumber
\end{eqnarray}
%
Here $ \tilde{F}$ denotes the transform of the horizontal variation, $F(x)$. 
The first term of the right hand side is simiular to Parker and Burton's equation (10) note.
Using the delay theorem of Laplace transforms on the appropriate partial fraction expansion (\citep{arfken2013mathematical})
the above may be inverse Laplace transformed. A subsequent Fourier inversion of horizontal wavenumber, $k$, yields
the flow response to one vertical mode of heating, $\phi_m(z)$:
%
\begin{eqnarray}
\label{equ_PB2}
w_m(x,t) & = & \frac{Q}{N^2} \left( 1 - \heavi( t-T) \right) F(x) \\ \nonumber
              & - & \frac{Q}{2N^2} \left(  F(x + c_m t) + F(x - c_m t) \right) \\ \nonumber
             & + & \frac{Q}{2 N^2} \heavi( t-T) \left( F(x-c_m (t-T)) \right) \\ \nonumber
            & + & \frac{Q}{2 N^2} \heavi( t-T) \left( F(x+ c_m (t-T))  \right). \\ \nonumber
\end{eqnarray}
% 
A few remarks are now appropriate. The above mode of response holds for any horizontal variation of sensible heating, it 
contains a horizontal phase speed determined by the vertical wavenumber $c_n$ and a response to steady heating may 
easily be obtained on setting $T \rightarrow \infty$, when terms with factor $\heavi(t-T)$ disappear. 

To proceed it is necessary to define the vertical variation of the thermal forcing, which is to be restricted to the troposphere as follows:
%
\begin{eqnarray}
\label{equ_heating_defn}
S_{(n-1)}(x,z,t) & = & Q F(x) \left( \heavi(t) - \heavi(t-T) \right) \\ \nonumber
           & \times & \sin \left( \frac{n \pi}{H_t} z \right)  (\heavi(z) - \heavi(z-H_t)), \quad \quad n \in \mathbb{Z}^+, \quad T>0.
\end{eqnarray}
%
Here $n$ is the number of half sinusoids of vertical variation of the sensible heating in the troposphere $0\leq z \leq H_t$,
which has $(n-1)$ tropospheric nodes. Note, the tropopause now coincides with the top of heating i.e. $S_{(n-1)} =0$, $z>H_t$.
Note also that the tropopause is taken to be identical with the assumed scale height of the troposphere (recall, 
the base state of density is $\rho(z) = \rho_s \exp \left( - \frac{z}{H_t} \right)$). 
Figure \ref{heating diagram} below is a schematic representation of the horizontal and vertical variation of the thermal forcing function 
we use throughout, plotted for $n=1$, corresponding to the gravest mode of heating across the tropopause. 
In this figure symbol $S_z (z) =  \sin \left( \frac{n \pi}{H_t} z \right) \left( \heavi(z) - \heavi(z-H_t) \right)$.
%
% 0
%
\begin{figure}[h!]
\caption{ Not to scale. Diagram of the horizontal and vertical heating variation of our
	      chosen sensible heating function. 
              The left (right) panel shows the vertical (horizontal) variation. 
              The horizontal variation is defined in equation \ref{equ_horizontal_heating}
              Note that, for the vertical variation, the distance, $z$, (altitude) co-ordinate is vertical. 
              The vertical variation depicted corresponds to the gravest mode of heating 
              $n=1$ in the troposphere, between the ground and the tropopause (broken red line). 
              In the right panel, the standard deviation of the normal distribution is $\sigma$ }
  \centering
    \includegraphics[width=1\textwidth]{heating_sketch.eps}
  \label{heating diagram}
\end{figure}
%
 
We project the truncated or confined heating defined in equation \ref{equ_heating_defn} 
over the eigenfunctions $\phi_m(z)$ 
%
\begin{equation}
\label{equ_1}
S_{(n-1)} (x,z,t) \approx Q F(x) \left( \heavi(t) - \heavi(t-T) \right) \sum_{i}^{M} s_i (n) \phi_i (z),
\end{equation}
%
where the Fourier coefficients $b_{m_z}(n)$ for the heating defined in equation \ref{equ_heating_defn} are:
%
\begin{equation}
b_{m_z} (n) = \frac{ 2  H_t } { \pi H } (-1)^{(n+1)} \sin \left( \frac{m_z \pi H_t }{ H } \right) \left( \frac{ n H^2 }{ n^2 H^2 - m_z^2 H_t^2 } \right).
\end{equation}
%
Here for example $n=1$ corresponds physically to the gravest mode of heating having zero nodes within the troposphere.
For the moment take $n=1$. $n=2$ corresponds to the next mode of heating identified by Nicholls as important, which will be considered when we
turn briefly to low-level cooling. 

Since the vertical structure equation for $w$ is linear, the response forced by the defined in equation \ref{equ_heating_defn}
may be written as a superposition of solutions \ref{equ_PB2}, weighted by the appropriate Fourier coefficient:
%
\begin{eqnarray}
\label{equ_2}
\frac{w }{Q} & = & \frac{1}{Q N^2} (1 - \heavi(t-T)) S_0(x,z,0) \\ \nonumber
& - & \frac{1}{2 N^2}  \sum_{m_z = 1}^{M} b_{m_z} (1)\left(F(x-ct)  + F(x+ct)   \right) \sin \left( \frac{m_z \pi} {H} z\right) \\ \nonumber
& + & \frac{1}{2 N^2} \heavi(t-T) \sum_{m_z = 1}^{M} b_{m_z} (1) F(x-c(t-T))  \sin \left( \frac{m_z \pi} {H} z\right), \\ \nonumber
& + & \frac{1}{2 N^2} \heavi(t-T) \sum_{m_z = 1}^{M} b_{m_z} (1) F(x+c(t-T))  \sin \left( \frac{m_z \pi} {H} z\right), \\ \nonumber
\end{eqnarray}
%
where of course:
%
\begin{equation}
c = \frac{NH}{\pi m_z}, \quad S_0 (x,z,0) = Q F(x) \sum_{m_z=1}^{M} b_{m_z} (1) \sin \left( \frac{m_z \pi} {H} z\right).
\end{equation}
%
Note that equation \ref{equ_1} may now be written as follows:
%
\begin{equation}
S_{0}(x,z,t) = (1 - \heavi(t-T) ) S_0(x,z,0).
 \end{equation}
%
%
%




%
%
%
\subsection{moved stuff}
...
Note that equations for field variables other than $w$ are not so suitable for the 
forthcoming application of Fourier and Laplace transform techniques (\citep{arfken2013mathematical}). Furthermore,
vertical motion is generally regarded as critical for the convection triggering and analysis of gravity waves 
which is the essential purpose of this study. 

...
Defining the horizontal variation of heating as:
%
\begin{equation}
\label{equ_horizontal_heating}
F(x) = \frac{1}{\sigma^\eta}\exp \left( -\frac{x^2}{2 \sigma^2} \right), \quad k =0,1,
\end{equation}
%
In the above, $\sigma$ is the width parameter for the horizontal variation of heating. Factor $\frac{1}{\sigma}$ 
ensures that total heating rate remains constant as $\sigma$ varies. We set switch parameter $\eta=0$ until further notice. 
Shortly we will define a more realistic, localised sensible heating. 




\end{document}